% This is "sig-alternate.tex" V2.0 May 2012
% This file should be compiled with V2.5 of "sig-alternate.cls" May 2012
%
% This example file demonstrates the use of the 'sig-alternate.cls'
% V2.5 LaTeX2e document class file. It is for those submitting
% articles to ACM Conference Proceedings WHO DO NOT WISH TO
% STRICTLY ADHERE TO THE SIGS (PUBS-BOARD-ENDORSED) STYLE.
% The 'sig-alternate.cls' file will produce a similar-looking,
% albeit, 'tighter' paper resulting in, invariably, fewer pages.
%
% ----------------------------------------------------------------------------------------------------------------
% This .tex file (and associated .cls V2.5) produces:
%       1) The Permission Statement
%       2) The Conference (location) Info information
%       3) The Copyright Line with ACM data
%       4) NO page numbers
%
% as against the acm_proc_article-sp.cls file which
% DOES NOT produce 1) thru' 3) above.
%
% Using 'sig-alternate.cls' you have control, however, from within
% the source .tex file, over both the CopyrightYear
% (defaulted to 200X) and the ACM Copyright Data
% (defaulted to X-XXXXX-XX-X/XX/XX).
% e.g.
% \CopyrightYear{2007} will cause 2007 to appear in the copyright line.
% \crdata{0-12345-67-8/90/12} will cause 0-12345-67-8/90/12 to appear in the copyright line.
%
% ---------------------------------------------------------------------------------------------------------------
% This .tex source is an example which *does* use
% the .bib file (from which the .bbl file % is produced).
% REMEMBER HOWEVER: After having produced the .bbl file,
% and prior to final submission, you *NEED* to 'insert'
% your .bbl file into your source .tex file so as to provide
% ONE 'self-contained' source file.
%
% ================= IF YOU HAVE QUESTIONS =======================
% Questions regarding the SIGS styles, SIGS policies and
% procedures, Conferences etc. should be sent to
% Adrienne Griscti (griscti@acm.org)
%
% Technical questions _only_ to
% Gerald Murray (murray@hq.acm.org)
% ===============================================================
%
% For tracking purposes - this is V2.0 - May 2012

\documentclass{sig-alternate}

%% customization goes into input macros
% set up tight list spacing
\usepackage{enumitem} 
\setlist{nolistsep,nosep}

% for toggles
\usepackage{etoolbox}

\newcommand {\studyquote}[1]{\em ``#1''\normalfont}

% CHANGE FROM TOGGLE TRUE TO TOGGLE FALSE FOR NON-ANONYMOUS RENDERING
% http://tex.stackexchange.com/questions/5894/latex-conditional-expression
\newtoggle{anonymous}
%\toggletrue{anonymous}
\togglefalse{anonymous}

% CHANGE FROM TOGGLE TRUE TO TOGGLE FALSE TO HIDE COMMENTS
\newtoggle{comments}
%\toggletrue{comments}
\toggletrue{comments}

% Comment region command (from Wesley Willett)
\usepackage[usenames]{color}
\usepackage[usenames,dvipsnames]{xcolor}
\iftoggle{comments} {
  %if we want to show comments
  %% ============================================================
  %% add your names here
  %% possible colors include:
  %%   Orange, magenta, NavyBlue, violet, BrickRed, OliveGreen
  %% ============================================================
  \newcommand {\ben}[1]{{\color{BrickRed}\bf{BZ: #1}\normalfont}}
  \newcommand {\yuxun}[1]{{\color{OliveGreen}\bf{YX: #1}\normalfont}}
}{
  %if we don't want to show comments
  \newcommand {\ben}[1]{}
  \newcommand {\yuxun}[1]{}
}

% \newcommand {\systemname}{HOBS }
% \newcommand {\systemnamenospace}{HOBS}


%%% Local Variables: 
%%% mode: latex
%%% TeX-master: "main"
%%% End: 



\begin{document}
%
% --- Author Metadata here ---
%\conferenceinfo{WOODSTOCK}{'97 El Paso, Texas USA}
%\CopyrightYear{2007} % Allows default copyright year (20XX) to be over-ridden - IF NEED BE.
%\crdata{0-12345-67-8/90/01}  % Allows default copyright data (0-89791-88-6/97/05) to be over-ridden - IF NEED BE.
% --- End of Author Metadata ---

\title{Which Device Knows Your Activity Better\\ Google Glass or Mobile Phones?}
% \subtitle{[Extended Abstract]
% \titlenote{A full version of this paper is available as
% \textit{Author's Guide to Preparing ACM SIG Proceedings Using
% \LaTeX$2_\epsilon$\ and BibTeX} at
% \texttt{www.acm.org/eaddress.htm}}}
%
% You need the command \numberofauthors to handle the 'placement
% and alignment' of the authors beneath the title.
%
% For aesthetic reasons, we recommend 'three authors at a time'
% i.e. three 'name/affiliation blocks' be placed beneath the title.
%
% NOTE: You are NOT restricted in how many 'rows' of
% "name/affiliations" may appear. We just ask that you restrict
% the number of 'columns' to three.
%
% Because of the available 'opening page real-estate'
% we ask you to refrain from putting more than six authors
% (two rows with three columns) beneath the article title.
% More than six makes the first-page appear very cluttered indeed.
%
% Use the \alignauthor commands to handle the names
% and affiliations for an 'aesthetic maximum' of six authors.
% Add names, affiliations, addresses for
% the seventh etc. author(s) as the argument for the
% \additionalauthors command.
% These 'additional authors' will be output/set for you
% without further effort on your part as the last section in
% the body of your article BEFORE References or any Appendices.

\numberofauthors{2} %  in this sample file, there are a *total*
% of EIGHT authors. SIX appear on the 'first-page' (for formatting
% reasons) and the remaining two appear in the \additionalauthors section.
%

\iftoggle{anonymous}{
  \author{
    \alignauthor 
    Anonymous for submission
  }
}{ %else
  \author{
    % You can go ahead and credit any number of authors here,
    % e.g. one 'row of three' or two rows (consisting of one row of three
    % and a second row of one, two or three).
    % 
    % The command \alignauthor (no curly braces needed) should
    % precede each author name, affiliation/snail-mail address and
    % e-mail address. Additionally, tag each line of
    % affiliation/address with \affaddr, and tag the
    % e-mail address with \email.
    \alignauthor Ben Zhang \\
    \affaddr{UC Berkeley EECS}\\
    \email{benzh@eecs.berkeley.edu}
    % 2nd. author
    \alignauthor Yuxun Zhou \\
    \affaddr{UC Berkeley EECS} \\
    \email{yuxun.zhou@eecs.berkeley.edu}
  }
}

\maketitle
\begin{abstract}
We believe that for activity recognition tasks, different devices will have different performance. However, it's unclear that 1) which device is better for which activity, 2) the quantitative differnece. In this project, we compare the activity recognition accuracy for two devices -- Google Glass and android phones. Through our extensive studies, we have found that Glass is better than phones in general, and with the same algorithm, Glass outperforms phone by XX\%.
\end{abstract}

% A category with the (minimum) three required fields
% \category{H.4}{Information Systems Applications}{Miscellaneous}
%A category including the fourth, optional field follows...
% \category{D.2.8}{Software Engineering}{Metrics}[complexity measures, performance measures]

% \terms{Theory}

\keywords{Activity Recognition; Google Glass; Android Phones; HMM}

%% The outline comes from this link: http://www.cosc.canterbury.ac.nz/open/students/peyton-jones-writing_a_paper-slides.pdf
\section{Introduction}
\label{sec:introduction}

Introduction usually contains two parts:
\begin{itemize}
\item Describe the problem. Using examples are recommended.
\item State your contributions. Bulleted list of contributions. Contributions should be refutable.
\end{itemize}


%%% Local Variables: 
%%% mode: latex
%%% TeX-master: "main"
%%% End: 

\section{Problem Formulation}
\label{sec:problem-formulation}

A supervised learning framework for activity recognition consists of data collection, transformation, and the design of adequate learning algorithm. In this section, we first briefly describe our data model (detailed system description and data collection is postponed to Section~\ref{sec:system-data} for completeness). In Section~\ref{subsec: data-transform}, we show how raw data is transformed into feature space. In addition, the problem of feature selection for efficient learning is discussed. In Section~\ref{subsec: learning}, we formulate the learning problem, and propose two methods for a comparative study.  

\subsection{Data Model}
\label{subsec: data-model}
We consider the sensor data model being a 3-dimensional accelerometer. Though gyroscope, magnetic sensor, rotation are also available and our data collection application supports them, for comparison with Pebble, we restrict the dataset to be accelerometer data. From a data analysis point of view, the obtained observations are a set of time series with fine time resolution. Each observation has 3-dimension $x, y, z$. 

We also label the data from a video recording we did when we collect the data. We consider five different activities \{standing, walking, downstairs, upstairs, running\} ({\bf for brevity, we will also use the number $\{1,2,...,5\}$ to represent each activity}). 

\subsection{Data Transformation \& Feature Selection}
\label{subsec: data-transform}

The raw data is a 3-dimensional time series exhibiting non-stationarity with fine time resolution. It is very difficult if not impossible to learn a good classifier that map the raw sensor readings directly to labels/classes. As most of the machine learning task, we face the challenge problem of feature extraction. In previous literature, it is widely acknowledged that a window frame based feature extraction could be adopted since most human activities are approximately periodic in nature. In this paper, we also consider feature extraction within a window frame, however, instead of merely focus on time and frequency domain feature, we proposed to add empirical distribution of frequencies as another categorical feature. This is because intuitively, different activities may induce very different frequency nature of the body. To be specific, we add binned distribution of Fast Fourier Transform (FFT) coefficients for each window as a $d$ dimensional feature, where $d$ is the number of bins. In addition, we calculate the empirical entropy as another feature candidate. As for time domain, we propose to add first order derivative (difference in time series) as an additional feature, the reason for this is the observation that different activities usually have distinguished ``rate of change'' in terms of body movement.

To sum up, we have the following features (also shown in Table~\ref{tab:feature1}):
\begin{itemize}
\item \textbf{Time Domain Features:} In each window, for one particular observation time series, we directly calculate Mean, Variance, and for each 3D streams of a particular sensor, we calculate the Covariance and the Magnitude. We also calculate the discrete first derivative and export the same statistics as features. 
\item \textbf{Frequency Domain Features:} We use Fast Fourier Transform to delve into frequency domain, and export energy and entropy of resulted FFT coefficients in each window.
\end{itemize}

\begin{table}
  \centering
  \begin{tabular}{c|c}
    \hline
    & $\bar{x}, \bar{y}, \bar{z}, var(x), var(y), var(z)$\\
    Time Domain  & $cov(x,y), cov(y,z), cov(z,x)$\\
    &$\sqrt{x^2+y^2+z^2}$\\
    & $x_{t+1}-x_{t}, y_{t+1}-y_{t}, z_{t+1}-z_{t}$\\
    \hline
    Frequency Domain  & $\sum_j w_j^2/N, -\sum_i p_ilog(p_i)$  \\
    \hline
  \end{tabular}
  \caption{Extracted Features}
  \label{tab:feature1}
\end{table}

Hence the transformed data set is a $20$ dimensional tuple, with $19$ features and $1$ label indicating the ground truth of corresponding activity. Note that another information that are implicitly collected is the transition between activities. In fact, the consideration of adding this information or not is going to lead us to two entirely different probabilistic models. Namely, if dynamic transition information is ignored, the problem reduces to common classification problem, and the underlying assumption is that each observation is draw $iid$ from some distribution. While if dynamic transitional information is incorporated in the model, because of the dependency of ``future'' and ``past'', usually local time series model should be considered for continuous measurement, and Markov Chains or semi Markov models should be used to model discrete hidden classes. Actually, the Hidden Markov Model (HMM) is one simple example of this kind. 

People would argue that incorporating additional information should yield better classification results. However, it may not be true in practice. On one hand, a dynamic model like HMM is more complex to train and may has higher parameter estimation error. On the other hand, it is arguable if human activity really exhibits Markov transitions. Granted that transition tendency exist, the transitional matrix may (or for most people)NOT be stationary,which rend HMM unsuitable because it requires stationary transitions to propagate. Of cause non-stationary model could be considered, but we doubt it will be too complex to carry out real parameter estimation and inference work. That's why in our project, we tried both multinomial logistic regression and HMM as machine learning tools for the purpose of activity recognition.  

Last but not least issue concerning the problem at hand is that the features we are using may not be all beneficial in terms of activity recognition. Since our previous feature selection is based mainly on empirical knowledge or intuition, the selected features may not be relevant or there may be redundancy in the feature space. Principle component analysis (PCA) could be used to eliminate redundancy, but transformed features after PCA do not have a clear meaning. In this project, we consider regularized method for statistical feature selection. Particularly, we applied $L1$ regularized logistic regression and check the solution path. 

\subsection{Activity Recognition as Supervised Learning}
\label{subsec: learning}

As an abstraction of above description, our task is to find a function $f: R^{19}\times T \rightarrow \{1,2,3,4,5\}$. If we assume that transition information is not important and consider obtained samples as $iid$ samples from certain distribution, our task is reduced to estimate $f: R^{19} \rightarrow \{1,2,3,4,5\}$. With this approximation, various classification methods are available, such as logistic regression (LR), support vector machine (SVM), linear discriminant analysis (LDA), quadratic discriminant analysis (QDA), decision tree (DT), etc. We adopt Multinomial Logistic Regression (MLR) and Hidden Markove Model (HMM) in this project.

%\begin{figure}[p]
%  \begin{center}
%    \includegraphics[width=0.4\textwidth]{C:/Dropbox/281A/figures/logiHmm.png}
%    \caption{Logistis Regression vs. HMM}
%    \label{fig:logiHmm}
%  \end{center}
%\end{figure}

Let's briefly review some key formulation for multinomial logistic regression and hidden markov model. A comprehensive introduction of the theory can be found in Prof.~Jordan's book.

As a generalization of logistic regress for multiclass problems, MLR models the probability of each class $y$ as a Logit function of linear combinations of feature variables $x$. Specifically, assume there are $K$ classes, for $m=1,2,...,K-1$
\begin{equation}
Pr(y_i=m) = \frac{e^{\beta_m^Tx_i}}{1+\sum_{k=1}^{K-1}\beta_k^Tx_i}
\end{equation}
and for $m=K$
\begin{equation}
Pr(y_i=K) = \frac{1}{1+\sum_{k=1}^{K-1}\beta_k^Tx_i}
\end{equation}
which ensures each probability has value between $[0,1]$ and their sum equal to $1$. The model parameters are vectors $\beta_1,\beta_2,...,\beta_{k-1}$. Note that our project, from a machine learning viewpoint consists of two task, which are parameter estimation for training and inference for testing. 

The parameter estimation for logistic regression is a widely studied topic. Usually, various algorithms are available to compute the ML estimation in non-regularized case and MAP estimation in $L2$ regularized case. In fact, for logistic regression without regularization, the ML problem just reads,
\begin{equation}
min_{\beta} \sum_i -logp(y_i|x_i,\beta)
\end{equation}
and with $L2$ norm regularization, we simply add a $L2$ penalty for $beta$,
\begin{equation}\nonumber
min_{\beta} \sum_i -logp(y_i|x_i,\beta) + \lambda \Vert \beta \Vert^2
\end{equation}
the penalty term can also be thought of putting a prior for $\beta$, thus sometimes is called MAP estimation. From a vast pool of such algorithms we find  generalized iterative scaling, Iteratively Reweighted Least Squares (IRLS), L-BFGS (gradient based nonlinear programming), and some specialized coordinate descent algorithms. However, $L2$ norm only regularize parameter ``proportionally'', and usually is depreciated when feature selection is our main purpose. As an alternative, $L1$ norm is used instead of $L2$ with advantages that it tend to push some parameters to be exactly zero, thus favors feature selection. Actually, the improvement here is very similar to the advantages of using LASSO. In our project, we perform both logistic regression without regularization, and with $L1$ regularization as an attempt for feature selection, i.e.
\begin{equation}\nonumber
min_{\beta} \sum_i -logp(y_i|x_i,\beta) + \lambda \Vert \beta \Vert_1
\end{equation}

The inference problem, or in other words, the prediction problem aims at estimate class label $y$ given a new set of observations. In the case of logistic regression, this is simply done by calculating class probability $Pr(y_i=$ based on estimated $\hat{\beta}$, and then choose a proper cut-off probability to assign class labels. 

As is mentioned in last section, a major drawback of common classification tools like Logistic Regression, SVM, QDA, Decision Tree make $iid$ assumption for samples and ignore temporal dependence in the phenomenon. Although most of existing research on activity recognition choose to ignore time dependence, we argue that transitional information may be able to help in that it can ``filter out'' impossible transitions and thus improve accuracy. The challenge for using HMM in our case is the choice of emission distribution. Because we have a relatively high feature dimension, some EDA should be performed before concluding any form of distribution $Pr(x|y)$. We summarized our choice in Table~\ref{table:distribution} and EDA will be provided in later chapters. Note that the choice is only approximately correct, especially for $var(x),var(y),var(z)$, a more proper choice could be matrix Gamma distribution, but the parameter estimation is hard. 
\begin{table}
\begin{center}
\begin{tabular}{c|c
}
      \hline
      $\bar{x},\bar{y},\bar{z}|y$& multivariate Gaussian\\
      \hline
      $var(x),var(y),var(z)|y$ &  multivariate Gaussian\\
      \hline
      $cov(x,y),cov(y,z),cov(z,x)|y$& multivariate Gaussian\\
      \hline
      $\sqrt{x^2+y^2+z^2}|y$& Gamma\\
      \hline
      $x_{t+1}-x_{t},y_{t+1}-y_{t},z_{t+1}-z_{t}|y$ & multivariate Gaussian\\
      \hline
      $\frac{\sum_j w_j^2}{N}|y$& Exponential \\ \hline
      $ -\sum_i p_ilog(p_i)|y$ & Gamma \\
  \hline
\end{tabular}
\end{center}
\caption{Extracted Features Distribution}
\label{table:distribution}
\end{table}


The parameter estimation for HMM in our case is fairly simple, in fact, since we are in a supervised learning case, while training the hidden states are actually observable. The log likelihood is already a ``complete'' likelihood. i.e. in the parameterization
\begin{equation} \nonumber
l(q,y) = log\left\{\pi_{q_0}\prod_{t=1}^{T-1} a_{q_t,q_{t+1}}\prod_{t=0}^T Pr(y_t|q_t,\eta)\right\}
\end{equation}
both $q$ and $y$ are known. The ML estimation can be calculated directly by taking derivatives,
\begin{equation} \nonumber
\hat{a}_{i,j} = \frac{m_{ij}}{\sum_{k=1}^M m_{ij}}
\end{equation}
with $m_{ij}$ defined as in the class, and the parameter estimate for distribution in table (\ref{table:distribution}) can be calculated trivially in each class. For Gaussian and exponential, the results are well known, as for Gamma distribution with parameter $\theta$ and $\alpha$,
\begin{equation}
\hat{\theta} = \frac{1}{\alpha N}\sum_{i=1}^Nx_i
\end{equation}
we can compute $\alpha$ iteratively 
\begin{equation}
\alpha \leftarrow \alpha - \frac{ \ln(\alpha) - \psi(\alpha) - s }{ \frac{1}{\alpha} - \psi^{\prime}(\alpha) }
\end{equation}
with 
\begin{equation} \nonumber
s = \ln{\left(\frac{1}{N}\sum_{i=1}^N x_i\right)} - \frac{1}{N}\sum_{i=1}^N\ln{(x_i)}
\end{equation}
After parameter estimation/training, we use the obtained model for inference/testing. The classification problem is equivalent to estimate hidden state $q$ give observation $y$ and the model parameters. The posterior probability of $q$ is defined as 
\begin{equation} \nonumber
p(q_t^i=1|y,\eta) = \gamma_t^i
\end{equation} 
where backward and forward propagation can be performed to find $\gamma_t^i$
\begin{equation}
\gamma (q_t) = \sum_{q_{t+1}} \frac{\alpha (q_t) a_{q_t,q_{t+1}}}{\sum_{q_t} \alpha (q_t)a_{q_t,q_{t+1}}} \gamma (q_{t+1})
\end{equation}
and
\begin{equation}
\alpha (q_{t+1}) = \sum_{q_t} \alpha (q_t)  a_{q_t,q_{t+1}} p(y_{t+1}|q_{t+1})
\end{equation}

Finally, we pick up a reasonable cut-off probability threshold, e.g. $[\frac{1}{5},...,\frac{1}{5}]$, to assign the class labels for each $q_t$.

%%% Local Variables: 
%%% mode: latex
%%% TeX-master: "main"
%%% End: 

\section{System and Data}
\label{sec:system-data}

In this section, we describe our system implementation -- Android platform and Pebble platform, followed by a briefly cover of our the experiement setup.

\subsection{Android Platform}
\label{sec:android-platform}

Our Android application is adapted from BearLoc\footnote{BearLoc is developed by Software Defined Buildings (SDB) group at Berkeley; the intial goal of that project is to provide an open-source implementation for indoor semantic localization service.}. We built a continous sensor monitoring application on top of \texttt{BearLocService} which makes sensor data collection

Our initial application involves sampling all sorts of sensor data from Android platform, including acclerometer, gyroscope, magnetic field, light, GPS, WiFi signals, etc. However, through some preliminary study, we have found that mobile phones and Google Glass cannot easily handle the extensive sensing tasks, especially when we are trying to sample acceleration in a relative high frequency. As has discussed in Section~\ref{subsec: data-model}, since we are interested in the comparison among these wearable devices and Pebble only has a 3D accelerometer, we then restricted our Android application only for sampling acclerometer data.

The sensor data is obtained through \texttt{onSensorChanged()} API provided by Android OS. The sampling frequency is resource adaptive -- it turns out to be 100Hz for phones and 50Hz for Glass. 

Once the data is measured, we record the time in millisecond accuracy by calling \texttt{System.currentTimeMillis()}. To extract the data, we logged the data locally by writing to a \texttt{csv} file and also post a HTTP request that saves the data to \texttt{MongoDB}.

\subsection{Pebble Platform}
\label{sec:pebble-platform}

The Pebble accelerometer data collection application is based on \texttt{AccelerometerService} API provided by Pebble OS. We register a callback function as the handler whenever accelerometer data is available. The callback function operates on \texttt{AccelData} which already has a field of timestamp in millisecond accuracy. 

The data storage and transmission is supported by \texttt{DataLogging} API which requires a customized Android application that uses Pebble SDK to retrieve logs by application universally unique identifier (UUID). We have also developed this accompany Android application for retrieving the data and writing them into a \texttt{csv} file for data analysis.

\subsection{Accelerometer Data}
\label{sec:accelerometer-axes}

On both Android and Pebble platform, the acclerometer coordinate system is defined relative to the device's screen (see Figure~\ref{fig:coordinate}). When the user is facing the screen, the axes are:
\begin{itemize}
\item x: horizontal and points to the right.
\item y: vertical and points up.
\item z: points towards the outside of of the screen face.
\end{itemize}

 
%% 04-20 17:40:46.392    4187-4187/name.benzhang.hellostep.app I/HelloStep﹕ {Sensor name="MPL Accelerometer", vendor="Invensense", version=1, type=1, maxRange=19.6133, resolution=0.039226603, power=0.0, minDelay=1000}

%% { "_id" : ObjectId("536f303fe0323929bee2cf88"), "vendor" : "STMicroelectronics", "power" : 0.23000000417232513, "min delay" : 20000, "resolution" : 0.01362034771591425, "sysnano" : NumberLong("170075408653778"), "epoch" : NumberLong("1399795756898"), "version" : 1, "max range" : 19.613300323486328, "type" : "sensor info", "model" : "KR3DM 3-axis Accelerometer", "sensor" : "accelerometer", "id" : "9026086e-bd07-3f96-9622-757da2907a93" }

However, the data differs in range, resolution and sample frequency. We use $g$ to denote the gravitational accelerometer constant (normal $9.81 m/s^2$). In Table~\ref{tab:sensorinfo}, we have listed the sensor information. Note that these values are extracted from our hardware sensor API so they are device dependent. Our window frame based feature extraction (see Section~\ref{subsec: data-transform}) is immune to the discrepancy of sampling frequency.

\begin{table}
  \centering
  \begin{tabular}{c|c|c|c}
    \hline
    Device & max range & resolution & sample frequency \\
    \hline
    Phone  & 19.6g     & 0.013g     & 100Hz  \\
    Glass  & 19.6g     & 0.039g     & 50Hz   \\
    Pebble & 4g        & 0.008g     & 25Hz   \\
    \hline
  \end{tabular}
  \caption{Sensor information of the devices.}
  \label{tab:sensorinfo}
\end{table}

\begin{figure}
  \centering
  \includegraphics[width=0.9\columnwidth]{figures/coordinates.png}
  \caption{3D co-ordinate system of our hardware platforms.}
  \label{fig:coordinate}
\end{figure}

\subsection{Data Collection}
\label{sec:data-collection-2}

With our data collection platform, we have performed a series of activities and collected the acclerometer data. Note that institutional review board (IRB) is a necessity for this kind of experiment if more volunteers are involved, thus for now the experiment is conducted by the authors. Figure~\ref{fig:exp} shows how these devices are worn on the author and the experiement is being conducted. To obtain the ground truth, we use an iPhone 5 to take videos continuously during the study. The video is manually processed to label the collected acclerometer data.

Two traces that we are using for this report were collected on May 10th. The author first walks out of a conference room, and then walk upstairs to 7th floor. He then pauses a while (standing) and then walk downstairs. After comning back to 4th floor, he passed the hallway by running. In such way we are able to collect all possible activities in one run. Notice that each activity duration is relatively random (some are quite short) which can be a challenge for a precise inference; however such randomness makes the data more real to people's normal life. 

\begin{figure}
  \centering
  \includegraphics[width=0.9\columnwidth]{figures/experiement_setup.png}
  \caption{Data collection process of simutaneously wearing three different devices and an illustration of their relative positions on the author's body.}
  \label{fig:exp}
\end{figure}



%%% Local Variables: 
%%% mode: latex
%%% TeX-master: "main"
%%% End: 


\section{Evaluation}

We collected several data sets, however in this project due to page limit, only two of them that is collected in Soda Hall are shown in this section.

\label{sec:evaluation}

\subsection{Raw Data and Extracted Features: EDA}

\subsection{Classification Results}

Both model fitting and testing errors are summarized in terms of overall accuracy and cross error table to indicate the performance of activity recognition with three different devices and two different learning methods. Let's examine the results in a device order for easy comparison.

\subsubsection{Google Glass Results}
\label{sec:glassresult}
\begin{itemize}
\item \textbf{With Multinomial Logistic Regression} \\
The results are shown in table (\ref{tab:glassLR1}) and (\ref{tab:glassLR2})
\begin{table}
\begin{center}
\begin{tabular}{|l|l|}
      \hline
      Training Error & Testing Error\\
      \hline
      $0.9036635$ & $0.7469244$ \\
      \hline
\end{tabular}
\caption{Google Glass with LR: overall accuracy}
\label{tab:glassLR1}
\end{center}
\end{table}

\begin{table}
\begin{center}
\begin{tabular}{|l|l|l|l|l|l|}
      \hline
      P/T& 1 & 2 &3 & 4 & 5 \\
      \hline
      1 &0.625&0.032&0.020&0.057&0.000\\
      2 &0.125&0.727&0.188&0.057&0.107\\
      3 &0.250&0.199&0.651&0.007&0.071\\
      4 &0.000&0.041&0.114&0.878&0.053\\
      5 & 0.000&0.000&0.026&0.000&0.767\\
      \hline
\end{tabular}
\caption{Google Glass with LR: Cross Error Table}
\label{tab:glassLR2}
\end{center}
\end{table}

\item \textbf{With HMM}

The results are shown in table (\ref{tab:glassHMM1}) and (\ref{tab:glassHMM2})
\begin{table}
\begin{center}
\begin{tabular}{|l|l|}
      \hline
      Training Error & Testing Error\\
      \hline
      $0.9599729$ & $0.7188049$ \\
      \hline
\end{tabular}
\caption{Google Glass with HMM: overall accuracy}
\label{tab:glassHMM1}
\end{center}
\end{table}
\begin{table}
\begin{center}
\begin{tabular}{|l|l|l|l|l|l|}
      \hline
      P/T& 1 & 2 &3 & 4 & 5 \\
      \hline
      1 &0.400&0.057&0.006&0.029&0.000\\
      2 &0.066&0.692&0.240&0.067&0.155\\
      3 &0.133&0.211&0.629&0.022&0.017\\
      4 &0.400&0.033&0.110&0.880&0.068\\
      5 & 0.000&0.005&0.013&0.000&0.759\\
      \hline
\end{tabular}
\caption{Google Glass with HMM: Cross Error Table}
\label{tab:glassHMM2}
\end{center}
\end{table}

\end{itemize}

\subsubsection{Android Phone Results} 
\label{subsec:phoneresult}
\begin{itemize}
\item \textbf{With Multinomial Logistic Regression} \\
The results are shown in table (\ref{tab:phoneLR1}) and (\ref{tab:phoneLR2})
\begin{table}
\begin{center}
\begin{tabular}{|l|l|}
      \hline
      Training Error & Testing Error\\
      \hline
      $0.748297$ & $0.3341772$ \\
      \hline
\end{tabular}
\caption{Phone with LR: overall accuracy}
\label{tab:phoneLR1}
\end{center}
\end{table}

\begin{table}
\begin{center}
\begin{tabular}{|l|l|l|l|l|l|}
      \hline
      P/T& 1 & 2 &3 & 4 & 5 \\
      \hline
      1 &0.125&0.019&0.071&0.017&0.000\\
      2 &0.125&0.593&0.313&0.366&0.318\\
      3 &0.000&0.196&0.306&0.241&0.181\\
      4 &0.125&0.129&0.282&0.294&0.272\\
      5 & 0.625&0.062&0.026&0.080&0.227\\
      \hline
\end{tabular}
\caption{Phone with LR: Cross Error Table}
\label{tab:phoneLR2}
\end{center}
\end{table}

\item \textbf{With HMM}

The results are shown in table (\ref{tab:phoneHMM1})and (\ref{tab:phoneHMM2})
\begin{table}
\begin{center}
\begin{tabular}{|l|l|}
      \hline
      Training Error & Testing Error\\
      \hline
      $0.9591281$ & $0.3578059$ \\
      \hline
\end{tabular}
\caption{Phone with HMM: overall accuracy}
\label{tab:phoneHMM1}
\end{center}
\end{table}
\begin{table}
\begin{center}
\begin{tabular}{|l|l|l|l|l|l|}
      \hline
      P/T& 1 & 2 &3 & 4 & 5 \\
      \hline
      1 &1&0.014&0.064&0.006&0.000\\
      2 &0&0.542&0.303&0.635&0.158\\
      3 &0&0.136&0.335&0.155&0.123\\
      4 &0&0.188&0.271&0.162&0.341\\
      5 & 0&0.117&0.026&0.040&0.376\\
      \hline
\end{tabular}
\caption{Phone with HMM: Cross Error Table}
\label{tab:phoneHMM2}
\end{center}
\end{table}

\end{itemize}

\subsubsection{Pebble Watch Results}
\label{subsec:pebbleresult}
\begin{itemize}
\item \textbf{With Multinomial Logistic Regression} \\
The results are shown in table (\ref{tab:pebbleLR1}) and (\ref{tab:pebbleLR2})
\begin{table}
\begin{center}
\begin{tabular}{|l|l|}
      \hline
      Training Error & Testing Error\\
      \hline
      $0.8658537$ & $0.6677966$ \\
      \hline
\end{tabular}
\caption{Pebble with LR: overall accuracy}
\label{tab:pebbleLR1}
\end{center}
\end{table}

\begin{table}
\begin{center}
\begin{tabular}{|l|l|l|l|l|l|}
      \hline
      P/T& 1 & 2 &3 & 4 & 5 \\
      \hline
      1 &0.166&0.065&0.000&0.013&0.000\\
      2 &0.166&0.631&0.274&0.083&0.242\\
      3 &0.000&0.163&0.709&0.111&0.060\\
      4 &0.166&0.131&0.000&0.763&0.090\\
      5 &0.500&0.0082&0.016&0.027&0.606\\
      \hline
\end{tabular}
\caption{Pebble with LR: Cross Error Table}
\label{tab:pebbleLR2}
\end{center}
\end{table}

\item \textbf{With HMM}

The results are shown in table (\ref{tab:pebbleHMM1})and (\ref{tab:pebbleHMM2})
\begin{table}
\begin{center}
\begin{tabular}{|l|l|}
      \hline
      Training Error & Testing Error\\
      \hline
      $0.9318902$ & $ 0.6237288$ \\
      \hline
\end{tabular}
\caption{Pebble with HMM: overall accuracy}
\label{tab:pebbleHMM1}
\end{center}
\end{table}
\begin{table}
\begin{center}
\begin{tabular}{|l|l|l|l|l|l|}
      \hline
      P/T& 1 & 2 &3 & 4 & 5 \\
      \hline
      1 &0.5&0.063&0.000&0.019&0.000\\
      2 &0.0&0.618&0.370&0.182&0.080\\
      3 &0.5&0.218&0.555&0.183&0.000\\
      4 &0.0&0.081&0.074&0.596&0.000\\
      5 &0.0&0.018&0.000&0.019&0.920\\
      \hline
\end{tabular}
\caption{Pebble with HMM: Cross Error Table}
\label{tab:pebbleHMM2}
\end{center}
\end{table}

\end{itemize}

%%% Local Variables: 
%%% mode: latex
%%% TeX-master: "main"
%%% End: 


\section{Related Work}
\label{sec:related-work}
\ben{This is just for adding the references. I have NOT read them well and summarized them well, so we need to change the texts.}

\cite{kwapisz2011activity} focuse on using the accelerometer on phones for activity recognition. \cite{yan2012energy, srinivasan2012accurate} described an adpative approach for energy efficient sampling. 
\cite{lee2011activity} uses Hierarchical Hidden Markov Models (HHMM). 
%%% Local Variables: 
%%% mode: latex
%%% TeX-master: "main"
%%% End: 


\section{Conclusion}
\label{sec:conclusion}
The general rule of writing paper: ``Start early. Very early''. Hastily-written papers get rejected. Papers are like wine: they need time to mature.

%%% Local Variables: 
%%% mode: latex
%%% TeX-master: "main"
%%% End: 


\iftoggle{anonymous}{
% no acks in anonymous submission
}{
  \section{Acknowledgments}
  Sincere thanks are delivered to Prof.~Michael Jordan.
}


% The following two commands are all you need in the
% initial runs of your .tex file to
% produce the bibliography for the citations in your paper.
\bibliographystyle{abbrv}
\bibliography{main}  % sigproc.bib is the name of the Bibliography in this case

% You must have a proper ".bib" file
%  and remember to run:
% latex bibtex latex latex
% to resolve all references
%
% ACM needs 'a single self-contained file'!
%
%APPENDICES are optional
%\balancecolumns
% \appendix
% %Appendix A
% \section{Headings in Appendices}
% The rules about hierarchical headings discussed above for
% the body of the article are different in the appendices.
% In the \textbf{appendix} environment, the command
% \textbf{section} is used to
% indicate the start of each Appendix, with alphabetic order
% designation (i.e. the first is A, the second B, etc.) and
% a title (if you include one).  So, if you need
% hierarchical structure
% \textit{within} an Appendix, start with \textbf{subsection} as the
% highest level. Here is an outline of the body of this
% document in Appendix-appropriate form:
% \subsection{Introduction}
% \subsection{The Body of the Paper}
% \subsubsection{Type Changes and  Special Characters}
% \subsubsection{Math Equations}
% \paragraph{Inline (In-text) Equations}
% \paragraph{Display Equations}
% \subsubsection{Citations}
% \subsubsection{Tables}
% \subsubsection{Figures}
% \subsubsection{Theorem-like Constructs}
% \subsubsection*{A Caveat for the \TeX\ Expert}
% \subsection{Conclusions}
% \subsection{Acknowledgments}
% \subsection{Additional Authors}
% This section is inserted by \LaTeX; you do not insert it.
% You just add the names and information in the
% \texttt{{\char'134}additionalauthors} command at the start
% of the document.

%\balancecolumns % GM June 2007
% That's all folks!
\end{document}

